\documentclass[11pt]{article}

% First load extension packages
\usepackage[a4paper,margin=25mm]{geometry}    % page layout
\usepackage{setspace} \onehalfspacing         % line spacing
\usepackage{amsfonts,amssymb,amsmath}         % useful math extensions
\usepackage{graphicx}                         % graphics import
\usepackage{siunitx}                          % easy SI units
\usepackage{color, colortbl}
\usepackage{longtable}

% Change paragraph indentation
\setlength{\parskip}{10pt}
\setlength{\parindent}{0pt}

% User-defined commands
\newcommand{\diff}[2]{\frac{\mathrm{d}{#1}}{\mathrm{d}{#2}}}
\newcommand{\ddiff}[2]{\frac{\mathrm{d}^2{#1}}{\mathrm{d}{#2}^2}}
\newcommand{\pdiff}[2]{\frac{\partial{#1}}{\partial{#2}}}
\newcommand{\pddiff}[2]{\frac{\partial^2{#1}}{\partial{#2}^2}}
\newcommand{\pdiffdiff}[3]{\frac{\partial^2{#1}}{\partial{#2}\partial{#3}}}
\renewcommand{\vec}[1]{\boldsymbol{#1}}
\newcommand{\Idx}{\;\mathrm{d}x}
\newcommand{\Real}{\mathbb{R}}
\newcommand{\Complex}{\mathbb{C}}
\newcommand{\Rational}{\mathbb{Q}}
\newcommand{\Integer}{\mathbb{Z}}
\newcommand{\Natural}{\mathbb{N}}
\definecolor{Gray}{gray}{0.9}

% topmatter
\title{Deep Learning methods in Genomic Medicine\\ Interim Report}

\author{Matthew Ramcharan \\ Supervised by Dr Colin\ Campbell}

\date{\today}

% main body
\begin{document}
\maketitle

\section{Introduction}

Somatic mutations are any alteration in cell that will not be passed onto future generations \cite{Griffiths2000}. A somatic mutation in a cell of a fully developed organism can have little to no noticeable effect on the organism itself (often leading to benign growths), however mutations that give rise to cancer are a special case. Cancer arises either from inactivation of tumor suppressor genes, or mutation of a special category of genes called proto-oncogenes, many of which regulate cell division. When mutated, proto-oncogenes enter a state of uncontrolled division and become Oncogenes, resulting in a cluster of cells called a tumor.  
These types of cell division lead to malignant tumors, in which the excessive cell proliferation causes the tumor to spread into surrounding tissues and cause damage. 

This means being capable of discriminating between benign and oncogenic mutations is integral to identifying cancer before the tumor gets too large, or in grows to be in a bad position to excise. 

In this project we focus on the prediction for somatic point mutation in the coding region of the human cancer genome. There are many cancer sequence databases currently being compiled, such as the Cancer Genome Atlas, COSMIC, 


\section{Literature review}
CScape was cool, lets make it better.
Gunnar Rasch has a great idea for genomes.

The problem of identifying which variations in genomic information drive disease is a well recorded and explored one \cite{Rogers2017,Quang2015,Shihab2013}. The specific case of if the disease driven is cancer
Rogers et al \cite{Rogers2017}

general-purpose pathogenic mutation classifiers across coding and noncoding regions.

\section{Project plan}

This project is primarily following this plan:

The datasets used
\begin{enumerate}
	\item CScape data \cite{Rogers2017} - Contains labelled genome data for 
	\item COSMIC Data - 
\end{enumerate}


The algorithms that will be used to predict these datasets are
\begin{enumerate}
	\item Support Vector Machine - The simplest kernel method for integrating different data sources is to combine the features from all sources into a single kernel \cite{Rogers2017}. Creates a binary classifier of either Pathogenic (Positive) or Control (Negative)
	\item Multiple Kernel Learning - A composite kernel is made from a set of base kernels, in which each base kernel is derived from an individual set of data, which are different features like Histone Modifications, 100-Way Sequence Conservation, or Genome Segmentation.
	\item Multi-task Multiple Kernel Learning - Applying the multiple kernel with multi-tasks where each task is a type of cancer (lung, breast, brain, etc)
\end{enumerate}



\begin{longtable}{ |p{5cm}|p{3cm}|p{6cm}|  }
	\hline
	Action & Timeframe & Project Relevance \\
	\hline\hline
	Research the problem & Weeks 2-4 & The dataset style used in  the CScape \cite{Rogers2017} and the method proposed in the Framework for Multi-task Multiple Kernel Learning \cite{Widmer2015} are integral to this project, and so should be understood well. \\
	\hline
 	Learn how to apply the Shogun toolkit for toy problems  & Weeks 4-5 & Learning how to use an existing implementation of the Multi-task Multiple Kernel Learning will provide greater insight into expected outputs when used on more complex datasets. \\
 	\hline
	Download and understand CScape dataset & Week 5 & Simplest form of dataset to be used in this project\\
	\hline	
 	Gain an understanding of the COSMIC \cite{Forbes2017} data & Weeks 5-6 & Selecting data to be used from the vast database COSMIC will be what is used for the main stage of this project - the implementation of Multi-task, multiple kernel learning. \\
 	\hline
 	 Understand existing models (CScape, FATHMM-MKL, & Weeks 6-9 &  \\
 	\hline
	\rowcolor{Gray}
 	Write Interim Report & Weeks 9-10 & \\
 	\hline
	  & Weeks 10-12 & . \\
 	\hline
 	 & Weeks 12 &  \\
	\hline
	& Christmas! & \\
 	\hline
	& Week 13 & \\
	\hline
	\rowcolor{Gray}
 	Ready for, and present Presentation & Week 14 & \\
 	\hline
	 & Weeks 14-16 &  \\
	\hline
	Finish Technical work, draw conclusions and consolidate report & Weeks 16-19 & Allow plenty of time to finish and polish the report. \\
	\hline
	\rowcolor{Gray}
	Draft of one chapter/section of final report hand-in. &  Week 17 & Section is written during consolidation period \\
	\hline
	Update/change report style based on feedback of single section and continue writing & Week 17-19 & Report changes are still during consolidation period\\
	\hline
	Create Poster & Week 19 & \\
	\hline
	Proofread Report & Week 19-20 &\\
	\hline
	\rowcolor{Gray}
	Submit full draft of final report and poster & Week 20 &\\
	\hline
	Update report and poster with any relevant details from Supervisor Dr Colin Campbell & Week 21 &\\
	\hline
	Proofread & Week 22 &\\
	\rowcolor{Gray}
	Final hand-in date for report and poster & Week 22.3 (First week of easter) & Noted for completeness, hand-in should be late Week 21/ early Week 22\\
	\hline
\end{longtable}

\section{Progress}

%Create the style, and include the bibliography.
\bibliographystyle{plain}
\bibliography{/home/matt/Documents/TechnicalProject/DeepLearningGenomicMed/Report/Literature/GenomicMedicine}

% the end
\end{document}
